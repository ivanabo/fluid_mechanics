\documentclass[a4paper,oneside,12pt]{memoir} % jednostrano: promijeniti twoside u oneside

% Paket inputenc omogucava direktno unosenje hrvatskih dijakritickih znakova 
% opcija utf8 za unicode (unix, linux, mac)
% opcija cp1250 za windowse
\usepackage[utf8]{inputenc}  % ukoliko se koristi XeLaTeX onda je \usepackage{xunicode}\usepackage{xltxtra}

% Stil za diplomski, unutra je ukljucena podrska za hrvatski jezik
\usepackage{diplomski}
% bibliografija na hrvatskom
\usepackage[languagenames,fixlanguage,croatian]{babelbib} % zahtijeva datoteku croatian.bdf
% hiperlinkovi 
\usepackage[pdftex]{hyperref} % ukoliko se koristi XeLaTeX onda je \usepackage[xetex]{hyperref}

% Odabir familije fontova:
% koristenjem XeLaTeX-a mogu se koristiti svi fontovi instalirani na racunalu, npr
% \defaultfontfeatures{Mapping=tex-text}
% \setmainfont[Ligatures={Common}]{Hoefler Text}
% ili
% \newcommand{\nas}[1]{\fontspec{Adobe Garamond Pro}\fontsize{24pt}{24pt}\color{Chocolate}\selectfont #1}
% i onda \nas{Naslov ...}
\usepackage{txfonts} % times new roman 

\usepackage{amsmath}

\newtheorem*{lema}{Lema}

% Paket graphicx sluzi za manipuliranje grafikom 
\usepackage[pdftex]{graphicx} % ukoliko se koristi XeLaTeX onda je \usepackage[xetex]{graphicx}
% Paket amsmath je vec ukljucen
% Dodatno definirane matematicke okoline:
% teorem (okolina: thm), lema (okolina: lem), korolar (okolina: cor),
% propozicija (okolina: prop), definicija (okolina: defn), napomena (okolina: rem),
% slutnja (okolina: conj), primjer (okolina: exa), dokaz (okolina: proof)
% Definirane su naredbe za ispisivanje skupova N, Z, Q, R i C
% Definirane su naredbe za funkcije koje se u hrvatskoj notaciji oznacavaju drukcije 
% nego u americkoj: tg, ctg, ... (\tgh za tangens hiperbolni)
% Takodjer su definirane naredbe za Ker i Im (da bi se razlikovala od naredbe za imaginarni dio kompleksnog
% broja, naredba se zove \slika).

\pagestyle{headings}
% uz paket fancyhdr mogu se lako kreirati fancy zaglavlja i podnozja
\pagestyle{empty}

\begin{document}

\subsubsection*{Mehanika fluida \\ Ivana Bobinac, Zadatak 3}	
$ $
\begin{lema}[Gronwall]
Neka su $\alpha, u \in C[a, b], \; \beta \in C^1[a, b], \; a, b \in \mathbb{R}, a < b$. Nadalje, neka su $\beta, u \geq 0$ t.d. vrijedi
\begin{equation} \label{eq:1}
u(t) \leq \beta(t) + \int_a^t \alpha(\tau) u(\tau) d\tau, \; t \in [a, b].
\end{equation}
Tada je
\begin{equation} \label{eq:2}
u(t) \leq \beta(a) exp\Big( \int_a^t \alpha(\tau) d\tau \Big) + \int_a^t \beta'(\tau) exp\Big( \int_{\tau}^t \alpha(z) dz \Big) d\tau, \; t \in [a, b].
\end{equation}
\end{lema}

\begin{proof}
Uočimo sljedeće:
\begin{align*}
- \int_a^t \beta(\tau) \frac{d}{d\tau} \Big( exp\Big( \int_{\tau}^t \alpha(r) dr \Big) \Big) d\tau &= (parcijalna \; integracija) \\
&= - \beta(\tau) exp\Big( \int_{\tau}^t \alpha(r) dr \Big) \Big|_{\tau=a}^{\tau=t} + \int_a^t \beta'(\tau) exp\Big( \int_{\tau}^t \alpha(r) dr \Big) d\tau \\
&= - \beta(t) + \beta(a) exp\Big( \int_a^t \alpha(r) dr \Big) + \int_a^t \beta'(\tau) exp\Big( \int_{\tau}^t \alpha(r) dr \Big) d\tau.
\end{align*}

S druge strane, za izraz na lijevoj strani vrijedi sljedeće:
\begin{equation*}
- \int_a^t \beta(\tau) \frac{d}{d\tau} \Big( exp\Big( \int_{\tau}^t \alpha(r) dr \Big) \Big) d\tau = \int_a^t \beta(\tau) exp \Big( \int_{\tau}^t \alpha(r) dr \Big) \alpha(\tau) d\tau.
\end{equation*}

Sada imamo:
\begin{equation*}
\beta(t) + \int_a^t \alpha(\tau) \beta(\tau) exp \Big( \int_{\tau}^t \alpha(r) dr \Big) d\tau = \beta(a) exp\Big( \int_a^t \alpha(r) dr \Big) + \int_a^t \beta'(\tau) exp\Big( \int_{\tau}^t \alpha(r) dr \Big) d\tau.
\end{equation*}

Pokažimo da (\ref{eq:1}) implicira sljedeću nejednakost:
\begin{equation} \label{eq:3}
u(t) \leq \beta(t) + \int_a^t \alpha(\tau) \beta(\tau) exp \Big( \int_{\tau}^t \alpha(r) dr \Big) d\tau.
\end{equation}
Time će biti dokazana tražena nejednakost (\ref{eq:2}). \newline
Definirajmo
\begin{equation}
v(s) = exp \Big( - \int_a^s \beta(r) dr \Big) \int_a^s \beta(r) u(r) dr, \; s \in [a, b].
\end{equation}
Deriviranjem tog izraza dobivamo
\begin{align*}
v'(s) &= - exp \Big( - \int_a^s \beta(r) dr \Big) \beta(s) \int_a^s \beta(r) u(r) dr + exp \Big( - \int_a^s \beta(r) dr \Big) \beta(s) u(s) \\
&= \underbrace{\Big( u(s) - \int_a^s \beta(r) u(r) dr \Big)}_{\leq \alpha(s)} \underbrace{\beta(s)}_{\geq 0} \underbrace{exp \Big( - \int_a^s \beta(r) dr \Big)}_{\geq 0} \\
&\leq \alpha(s) \beta(s) exp \Big( - \int_a^s \beta(r) dr \Big).
\end{align*}
Sada integriranjem od $a$ do $t$ dobivamo
\begin{equation}
\int_a^t v'(s) ds \leq \int_a^t \alpha(s) \beta(s) exp \Big( - \int_a^s \beta(r) dr \Big) ds.
\end{equation}
Lijeva strana je jednaka $v(t) - v(a)$, no iz definicije od $v$ vidimo da je $v(a) = 0$ pa zapravo imamo ocjenu za $v(t)$:
\begin{equation} \label{eq:6}
v(t) \leq \int_a^t \alpha(s) \beta(s) exp \Big( - \int_a^s \beta(r) dr \Big) ds.
\end{equation}
Definicija od $v$ također očito implicira jednakost
\begin{equation} \label{eq:7}
\int_a^t \beta(r) u(r) dr = exp \Big( \int_a^t \beta(r) dr \Big) v(t).
\end{equation}

(\ref{eq:6}) i (\ref{eq:7}) skupa daju
\begin{align*}
\int_a^t \beta(r) u(r) dr &\leq exp \Big( \int_a^t \beta(r) dr \Big) \int_a^t \alpha(s) \beta(s) exp \Big( - \int_a^s \beta(r) dr \Big) ds \\
&= \int_a^t \alpha(s) \beta(s) exp \Big( \int_a^t \beta(r) dr - \int_a^s \beta(r) dr \Big) ds \\
&= \int_a^t \alpha(s) \beta(s) exp \Big( \int_s^t \beta(r) dr \Big) ds 
\end{align*}

Sada primjenom ove posljednje ocjene u (\ref{eq:1}) slijedi upravo (\ref{eq:3}).
\end{proof}
\end{document}