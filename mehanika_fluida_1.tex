\documentclass[a4paper,oneside,12pt]{memoir} % jednostrano: promijeniti twoside u oneside

% Paket inputenc omogucava direktno unosenje hrvatskih dijakritickih znakova 
% opcija utf8 za unicode (unix, linux, mac)
% opcija cp1250 za windowse
\usepackage[utf8]{inputenc}  % ukoliko se koristi XeLaTeX onda je \usepackage{xunicode}\usepackage{xltxtra}

% Stil za diplomski, unutra je ukljucena podrska za hrvatski jezik
\usepackage{diplomski}
% bibliografija na hrvatskom
\usepackage[languagenames,fixlanguage,croatian]{babelbib} % zahtijeva datoteku croatian.bdf
% hiperlinkovi 
\usepackage[pdftex]{hyperref} % ukoliko se koristi XeLaTeX onda je \usepackage[xetex]{hyperref}

% Odabir familije fontova:
% koristenjem XeLaTeX-a mogu se koristiti svi fontovi instalirani na racunalu, npr
% \defaultfontfeatures{Mapping=tex-text}
% \setmainfont[Ligatures={Common}]{Hoefler Text}
% ili
% \newcommand{\nas}[1]{\fontspec{Adobe Garamond Pro}\fontsize{24pt}{24pt}\color{Chocolate}\selectfont #1}
% i onda \nas{Naslov ...}
\usepackage{txfonts} % times new roman 

\usepackage{amsmath}

% Paket graphicx sluzi za manipuliranje grafikom 
\usepackage[pdftex]{graphicx} % ukoliko se koristi XeLaTeX onda je \usepackage[xetex]{graphicx}
% Paket amsmath je vec ukljucen
% Dodatno definirane matematicke okoline:
% teorem (okolina: thm), lema (okolina: lem), korolar (okolina: cor),
% propozicija (okolina: prop), definicija (okolina: defn), napomena (okolina: rem),
% slutnja (okolina: conj), primjer (okolina: exa), dokaz (okolina: proof)
% Definirane su naredbe za ispisivanje skupova N, Z, Q, R i C
% Definirane su naredbe za funkcije koje se u hrvatskoj notaciji oznacavaju drukcije 
% nego u americkoj: tg, ctg, ... (\tgh za tangens hiperbolni)
% Takodjer su definirane naredbe za Ker i Im (da bi se razlikovala od naredbe za imaginarni dio kompleksnog
% broja, naredba se zove \slika).

\pagestyle{headings}
% uz paket fancyhdr mogu se lako kreirati fancy zaglavlja i podnozja
\pagestyle{empty}

\begin{document}

\subsubsection*{Mehanika fluida \\ Ivana Bobinac, Zadatak 1}	
$\;\;\;\;\;$Jednadžba kontinuiteta:
\begin{equation}
\partial_t \rho + div(\rho v) = 0
\end{equation}

Jednadžbe gibanja:
\begin{align}
\rho (\partial_t v + (\nabla v)v) &= divT + f \\
T &= T^T
\end{align}

Pretpostavke: homogenost, izotropnost. \newline

Stress-strain tenzor: $T = -p I + \lambda (div v) I + 2 \mu sym\nabla v$

\begin{flalign*}
\rho &:\Omega^{\phi} x \; \mathbb{R} \rightarrow \mathbb{R}^3 \; gustoca&&\\
 v &:\Omega^{\phi} x \; \mathbb{R} \rightarrow \mathbb{R}^3 \; brzina&&\\
 f &:\Omega^{\phi} x \; \mathbb{R} \rightarrow \mathbb{R}^3 \; gustoca \; vanjske \; volumne \; sile&&\\
 p &:\Omega^{\phi} x \; \mathbb{R} \rightarrow \mathbb{R} \; pritisak \; (tlak)&&\\
\lambda, & \; \mu \; konstante
\end{flalign*}

Raspišimo
\begin{align*}
divT &= div(-pI+\lambda (divv)I+2\mu sym\nabla v) \\
     &= div(-pI)+div(\lambda (divv)I)+div(2\mu sym\nabla v)
\end{align*}

Prvo imamo:
\[
-pI
=
\begin{bmatrix}
    -p &  0 &  0 \\
     0 & -p &  0 \\
     0 &  0 & -p
\end{bmatrix}
\]

pa je

\[
div(-pI)
=
\begin{bmatrix}
    -\frac{\partial p}{\partial x_1} + 0 + 0 \\
     0 -\frac{\partial p}{\partial x_2} + 0 \\
     0 + 0 -\frac{\partial p}{\partial x_3}
\end{bmatrix}
=
\begin{bmatrix}
    -\frac{\partial p}{\partial x_1} \\
    -\frac{\partial p}{\partial x_2} \\
    -\frac{\partial p}{\partial x_3}
\end{bmatrix}
=
-\nabla p.
\]

Vrijedi
\[
divv
=
tr(\nabla v)
= tr(
\begin{bmatrix}
    \frac{\partial v_1}{\partial x_1} &  \frac{\partial v_1}{\partial x_2} &  \frac{\partial v_1}{\partial x_3} \\
    \frac{\partial v_2}{\partial x_1} &  \frac{\partial v_2}{\partial x_2} &  \frac{\partial v_2}{\partial x_3} \\
    \frac{\partial v_3}{\partial x_1} &  \frac{\partial v_3}{\partial x_2} &  \frac{\partial v_3}{\partial x_3}
\end{bmatrix})
=
\sum_{i=1}^{3} \frac{\partial v_i}{\partial x_i}
\]

pa je
\[
div(\lambda divvI)
=
\lambda div(
\begin{bmatrix}
     \sum_{i=1}^{3} \frac{\partial v_i}{\partial x_i} &  0 &  0 \\
     0 &  \sum_{i=1}^{3} \frac{\partial v_i}{\partial x_i} &  0 \\
     0 &  0 &  \sum_{i=1}^{3} \frac{\partial v_i}{\partial x_i}
\end{bmatrix})
=
\lambda \begin{bmatrix}
     \frac{\partial}{\partial x_1} (\sum_{i=1}^{3} \frac{\partial v_i}{\partial x_i}) \\
     \frac{\partial}{\partial x_2} (\sum_{i=1}^{3} \frac{\partial v_i}{\partial x_i}) \\
     \frac{\partial}{\partial x_3} (\sum_{i=1}^{3} \frac{\partial v_i}{\partial x_i})
\end{bmatrix}
\]

Za zadnji član imamo:
\begin{equation*}
sym \nabla v = \frac{\nabla v + (\nabla v)^T}{2} \implies 2 \mu sym \nabla v = \mu (\nabla v + (\nabla v)^T).
\end{equation*}

Stoga je
\begin{equation*}
div(2\mu sym\nabla v) = \mu div(\nabla v + (\nabla v)^T).
\end{equation*}

Vrijedi
\[
\nabla v + (\nabla v)^T
=
\begin{bmatrix}
    \frac{\partial v_1}{\partial x_1} &  \frac{\partial v_1}{\partial x_2} &  \frac{\partial v_1}{\partial x_3} \\
    \frac{\partial v_2}{\partial x_1} &  \frac{\partial v_2}{\partial x_2} &  \frac{\partial v_2}{\partial x_3} \\
    \frac{\partial v_3}{\partial x_1} &  \frac{\partial v_3}{\partial x_2} &  \frac{\partial v_3}{\partial x_3}
\end{bmatrix}
+
\begin{bmatrix}
    \frac{\partial v_1}{\partial x_1} &  \frac{\partial v_2}{\partial x_1} &  \frac{\partial v_3}{\partial x_1} \\
    \frac{\partial v_1}{\partial x_2} &  \frac{\partial v_2}{\partial x_2} &  \frac{\partial v_3}{\partial x_2} \\
    \frac{\partial v_1}{\partial x_3} &  \frac{\partial v_2}{\partial x_3} &  \frac{\partial v_3}{\partial x_3}
\end{bmatrix}
=
\begin{bmatrix}
    2 \frac{\partial v_1}{\partial x_1} &  \frac{\partial v_1}{\partial x_2}+\frac{\partial v_2}{\partial x_1} &  \frac{\partial v_1}{\partial x_3}+\frac{\partial v_3}{\partial x_1} \\
    \frac{\partial v_2}{\partial x_1}+\frac{\partial v_1}{\partial x_2} &  2 \frac{\partial v_2}{\partial x_2} &  \frac{\partial v_2}{\partial x_3}+\frac{\partial v_3}{\partial x_2} \\
    \frac{\partial v_3}{\partial x_1}+\frac{\partial v_1}{\partial x_3} &  \frac{\partial v_3}{\partial x_2}+\frac{\partial v_2}{\partial x_3} &  2 \frac{\partial v_3}{\partial x_3}
\end{bmatrix}
\]

pa imamo
\[
\mu div(\nabla v + (\nabla v)^T)
= \mu
\begin{bmatrix}
    \frac{\partial}{\partial x_1} (2 \frac{\partial v_1}{\partial x_1}) + \frac{\partial}{\partial x_2} (\frac{\partial v_1}{\partial x_2}+\frac{\partial v_2}{\partial x_1}) + \frac{\partial}{\partial x_3} (\frac{\partial v_1}{\partial x_3}+\frac{\partial v_3}{\partial x_1}) \\
    \frac{\partial}{\partial x_1} (\frac{\partial v_2}{\partial x_1}+\frac{\partial v_1}{\partial x_2}) + \frac{\partial}{\partial x_2} (2 \frac{\partial v_2}{\partial x_2}) + \frac{\partial}{\partial x_3} (\frac{\partial v_2}{\partial x_3}+\frac{\partial v_3}{\partial x_2}) \\
    \frac{\partial}{\partial x_1} (\frac{\partial v_3}{\partial x_1}+\frac{\partial v_1}{\partial x_3}) + \frac{\partial}{\partial x_2} (\frac{\partial v_3}{\partial x_2}+\frac{\partial v_2}{\partial x_3}) + \frac{\partial}{\partial x_3} (2 \frac{\partial v_3}{\partial x_3})
\end{bmatrix}.
\]

Sve skupa imamo
\[
divT
=
\begin{bmatrix}
    -\frac{\partial p}{\partial x_1} \\
    -\frac{\partial p}{\partial x_2} \\
    -\frac{\partial p}{\partial x_3}
\end{bmatrix} + 
\lambda \begin{bmatrix}
     \frac{\partial}{\partial x_1} (\sum_{i=1}^{3} \frac{\partial v_i}{\partial x_i}) \\
     \frac{\partial}{\partial x_2} (\sum_{i=1}^{3} \frac{\partial v_i}{\partial x_i}) \\
     \frac{\partial}{\partial x_3} (\sum_{i=1}^{3} \frac{\partial v_i}{\partial x_i})
\end{bmatrix} + \mu
\begin{bmatrix}
    \frac{\partial}{\partial x_1} (2 \frac{\partial v_1}{\partial x_1}) + \frac{\partial}{\partial x_2} (\frac{\partial v_1}{\partial x_2}+\frac{\partial v_2}{\partial x_1}) + \frac{\partial}{\partial x_3} (\frac{\partial v_1}{\partial x_3}+\frac{\partial v_3}{\partial x_1}) \\
    \frac{\partial}{\partial x_1} (\frac{\partial v_2}{\partial x_1}+\frac{\partial v_1}{\partial x_2}) + \frac{\partial}{\partial x_2} (2 \frac{\partial v_2}{\partial x_2}) + \frac{\partial}{\partial x_3} (\frac{\partial v_2}{\partial x_3}+\frac{\partial v_3}{\partial x_2}) \\
    \frac{\partial}{\partial x_1} (\frac{\partial v_3}{\partial x_1}+\frac{\partial v_1}{\partial x_3}) + \frac{\partial}{\partial x_2} (\frac{\partial v_3}{\partial x_2}+\frac{\partial v_2}{\partial x_3}) + \frac{\partial}{\partial x_3} (2 \frac{\partial v_3}{\partial x_3})
\end{bmatrix}
\]

odnosno
\[
divT
=
\begin{bmatrix}
    -\partial_1 p + (\lambda + \mu) (\partial_{11} v_1 + \partial_{21} v_2 + \partial_{31} v_3) + \mu (\partial_{11} v_1 + \partial_{22} v_1 + \partial_{33} v_1) \\
    -\partial_2 p + (\lambda + \mu) (\partial_{12} v_1 + \partial_{22} v_2 + \partial_{32} v_3) + \mu (\partial_{11} v_2 + \partial_{22} v_2 + \partial_{33} v_2) \\
    -\partial_3 p + (\lambda + \mu) (\partial_{13} v_1 + \partial_{23} v_2 + \partial_{33} v_3) + \mu (\partial_{11} v_3 + \partial_{22} v_3 + \partial_{33} v_3)
\end{bmatrix}.
\]


Jednadžba (2) sada glasi
\[
\rho
\begin{bmatrix}
\partial_t v_1 + (\partial_1 v_1)v_1 + (\partial_2 v_1)v_2 + (\partial_3 v_1)v_3 \\
\partial_t v_2 + (\partial_1 v_2)v_1 + (\partial_2 v_2)v_2 + (\partial_3 v_2)v_3 \\
\partial_t v_3 + (\partial_1 v_3)v_1 + (\partial_2 v_3)v_2 + (\partial_3 v_3)v_3
\end{bmatrix}
=
\]
\[ =
\begin{bmatrix}
    -\partial_1 p + (\lambda + \mu) (\partial_{11} v_1 + \partial_{21} v_2 + \partial_{31} v_3) + \mu (\partial_{11} v_1 + \partial_{22} v_1 + \partial_{33} v_1) + f_1 \\
    -\partial_2 p + (\lambda + \mu) (\partial_{12} v_1 + \partial_{22} v_2 + \partial_{32} v_3) + \mu (\partial_{11} v_2 + \partial_{22} v_2 + \partial_{33} v_2) + f_2\\
    -\partial_3 p + (\lambda + \mu) (\partial_{13} v_1 + \partial_{23} v_2 + \partial_{33} v_3) + \mu (\partial_{11} v_3 + \partial_{22} v_3 + \partial_{33} v_3) + f_3
\end{bmatrix}
\]



\end{document}