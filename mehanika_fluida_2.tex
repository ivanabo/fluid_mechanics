\documentclass[a4paper,oneside,12pt]{memoir} % jednostrano: promijeniti twoside u oneside

% Paket inputenc omogucava direktno unosenje hrvatskih dijakritickih znakova 
% opcija utf8 za unicode (unix, linux, mac)
% opcija cp1250 za windowse
\usepackage[utf8]{inputenc}  % ukoliko se koristi XeLaTeX onda je \usepackage{xunicode}\usepackage{xltxtra}

% Stil za diplomski, unutra je ukljucena podrska za hrvatski jezik
\usepackage{diplomski}
% bibliografija na hrvatskom
\usepackage[languagenames,fixlanguage,croatian]{babelbib} % zahtijeva datoteku croatian.bdf
% hiperlinkovi 
\usepackage[pdftex]{hyperref} % ukoliko se koristi XeLaTeX onda je \usepackage[xetex]{hyperref}

% Odabir familije fontova:
% koristenjem XeLaTeX-a mogu se koristiti svi fontovi instalirani na racunalu, npr
% \defaultfontfeatures{Mapping=tex-text}
% \setmainfont[Ligatures={Common}]{Hoefler Text}
% ili
% \newcommand{\nas}[1]{\fontspec{Adobe Garamond Pro}\fontsize{24pt}{24pt}\color{Chocolate}\selectfont #1}
% i onda \nas{Naslov ...}
\usepackage{txfonts} % times new roman 

\usepackage{amsmath}

% Paket graphicx sluzi za manipuliranje grafikom 
\usepackage[pdftex]{graphicx} % ukoliko se koristi XeLaTeX onda je \usepackage[xetex]{graphicx}
% Paket amsmath je vec ukljucen
% Dodatno definirane matematicke okoline:
% teorem (okolina: thm), lema (okolina: lem), korolar (okolina: cor),
% propozicija (okolina: prop), definicija (okolina: defn), napomena (okolina: rem),
% slutnja (okolina: conj), primjer (okolina: exa), dokaz (okolina: proof)
% Definirane su naredbe za ispisivanje skupova N, Z, Q, R i C
% Definirane su naredbe za funkcije koje se u hrvatskoj notaciji oznacavaju drukcije 
% nego u americkoj: tg, ctg, ... (\tgh za tangens hiperbolni)
% Takodjer su definirane naredbe za Ker i Im (da bi se razlikovala od naredbe za imaginarni dio kompleksnog
% broja, naredba se zove \slika).

\pagestyle{headings}
% uz paket fancyhdr mogu se lako kreirati fancy zaglavlja i podnozja
\pagestyle{empty}

\begin{document}

\subsubsection*{Mehanika fluida \\ Ivana Bobinac, Zadatak 2}	
$ $
\textbf{Zadana je funkcija $V$ na $\mathcal{U} = \mathbb{R} \backslash \{0\}$ sa $V(r,\varphi) = \frac{1}{2 \pi} \varphi$. Izračunajmo $\nabla V$.}  \\
Polarne koordinate su dane sa:
\begin{align*}
x_1 &= r cos \varphi,\\
x_2 &= r sin \varphi.
\end{align*}

Kako je
\begin{align*}
r &= \sqrt{x_1^2 + x_2^2}, \\
\varphi &= arctg\frac{x_2}{x_1},
\end{align*}
funkcija $V$ je u Kartezijevim koordinatama dana sa 
\begin{equation*}
V(x_1,x_2) = \frac{1}{2 \pi} arctg\frac{x_2}{x_1}.
\end{equation*}

Računamo:
\begin{align*}
\partial_1 V(x_1,x_2) &= \frac{1}{2 \pi} \frac{1}{1+\frac{x_2}{x_1}^2}\frac{-x_2}{x_1^2} = \frac{1}{2 \pi} \frac{-x_2}{x_1^2 + x_2^2},\\
\partial_2 V(x_1,x_2) &= \frac{1}{2 \pi} \frac{1}{1+\frac{x_2}{x_1}^2}\frac{1}{x_1} = \frac{1}{2 \pi} \frac{1}{\frac{x_1^2 + x_2^2}{x_1}} = \frac{1}{2 \pi} \frac{x_1}{x_1^2 + x_2^2}.
\end{align*}

Vraćanjem na polarne koordinate imamo
\begin{align*}
\partial_1 V(r,\varphi) &= \frac{1}{2 \pi} \frac{-r sin\varphi}{r^2} = \frac{1}{2 \pi} \frac{1}{r}(-sin\varphi),\\
\partial_2 V(r,\varphi) &= \frac{1}{2 \pi} \frac{r cos\varphi}{r^2} = \frac{1}{2 \pi} \frac{1}{r}(cos\varphi).
\end{align*}

Dakle, sve skupa dobivamo traženo:
\begin{equation*}
\nabla V(r,\varphi) = \frac{1}{2 \pi} \frac{1}{r} \begin{bmatrix}
   												      -sin \varphi \\
 											           cos \varphi
  											      \end{bmatrix}.
\end{equation*}
$ $\\
$ $\\
\textbf{Izračunajmo protok i cirkulaciju od $v:=\nabla V$ na kružnici radijusa $R$ sa središtem u ishodištu.} Protok i cirkulacija su, po definiciji, redom:
\begin{align*}
protok &= I_P = \int_\Gamma v \cdot n d\gamma \\
cirkulacija &= I_C = \int_\Gamma v \cdot \tau d\gamma
\end{align*}
Dakle, zadana krivulja $\Gamma$ je $\partial K(0,R)$  koju parametriziramo sa \\
$\gamma : [0, 2\pi] \rightarrow \mathbb{R}^2$, \; $\gamma(t) = (Rcost,\; Rsint)$. Vrijedi: $\gamma'(t) = (-Rsint, \; Rcost)$ te $\Vert \gamma'(t) \Vert = R$.
Zapišimo $v$ u Kartezijevim koordinatama:
\begin{equation*}
v(x_1,x_2) = \frac{1}{2\pi}\frac{1}{x_1^2+x_2^2} \begin{bmatrix}
   												      -x_2 \\
 											           x_1
  											      \end{bmatrix}.
\end{equation*}
Sada imamo:
\begin{equation*}
v(\gamma(t)) = v(Rcost, Rsint) = \frac{1}{2\pi}\frac{1}{R^2} \begin{bmatrix}
   												  		       -Rsint \\
 											      		        Rcost
  											    		     \end{bmatrix} 
  							   = \frac{1}{2\pi}\frac{1}{R} \begin{bmatrix}
   												     		   -sint \\
 											        		    cost
  											   			   \end{bmatrix}.
\end{equation*}
Jedinična tangenta je dana sa:
\begin{equation*}
\tau = \frac{\gamma'(t)}{\Vert \gamma'(t) \Vert} = \begin{bmatrix}
   												       -sint \\
 											        	cost
  											   		\end{bmatrix},
\end{equation*}
a normala sa:
\begin{equation*}
n = \tau \; x \; e_3 = \begin{bmatrix}
   					 cost \\
 					 sint
  				 \end{bmatrix}.
\end{equation*}
Sada imamo:
\begin{equation*}
I_C = \int_\Gamma v \cdot \tau d\gamma = \int_0^{2\pi} v(\gamma(t)) \cdot \frac{\gamma'(t)}{\Vert \gamma'(t) \Vert} \Vert \gamma'(t) \Vert dt = \int_0^{2\pi} v(\gamma(t)) \cdot \gamma'(t)  dt.
\end{equation*}
Kako je
\begin{equation*}
v(\gamma(t)) \cdot \gamma'(t) = \frac{1}{2\pi}\frac{1}{R} R \begin{bmatrix}
   												     		   -sint \\
 											        		    cost
  											   			   \end{bmatrix} 
  											   \cdot 	   \begin{bmatrix}
   												     		   -sint \\
 											        		    cost
  											   			   \end{bmatrix}
  							  = \frac{1}{2\pi} (sin^2t + cos^2t) = \frac{1}{2\pi},
\end{equation*}
konačno imamo
\begin{equation*}
I_C = \int_0^{2\pi} \frac{1}{2\pi} dt = 1.
\end{equation*}
Za protok, računamo
\begin{equation*}
v(\gamma(t)) \cdot n = \frac{1}{2\pi}\frac{1}{R} \begin{bmatrix}
   												     -sint \\
 											          cost
  											   	 \end{bmatrix} 
  											   \cdot 	   \begin{bmatrix}
   												     		    cost \\
 											        		    sint
  											   			   \end{bmatrix}
  					 = \frac{1}{2\pi}\frac{1}{R} (-sintcost + costsint) = 0
\end{equation*}
pa je
\begin{equation*}
I_P = \int_0^{2\pi} 0 dt = 0.
\end{equation*}
$ $\\
$ $\\
\textbf{Pokažimo još da vrijedi: $div v = div \nabla V = \Delta V = 0$.} \\
Ranije smo izračunali parcijalne derivacije ove funkcije pa derivirajući parcijalno te izraze još jednom po istim varijablama dobivamo:
\begin{align*}
\partial_{11} V(x_1,x_2) &= \frac{1}{2 \pi} \frac{2x_1 x_2}{(x_1^2+x_2^2)^2} = \frac{1}{\pi} \frac{x_1 x_2}{(x_1^2 + x_2^2)^2},\\
\partial_{22} V(x_1,x_2) &= \frac{1}{2 \pi} \frac{-x_1 2x_2}{(x_1^2+x_2^2)^2} = \frac{1}{\pi} \frac{-x_1 x_2}{(x_1^2 + x_2^2)^2}.
\end{align*}
Sada je očito $\Delta V = \partial_{11} V + \partial_{22} V = 0$.
$ $\\
$ $\\
$ $\\
$ $\\
\textbf{Pokažimo sljedeće:\\
Neka je $f:\mathbb{R}^3 \rightarrow \mathbb{R}$ dovoljno glatka. Tada je $rot \nabla f = 0$.}\\
$ $\\
Vrijedi $\nabla f = (\partial_1 f,\partial_2 f,\partial_3 f)$.\\
Rotacija vektorske funkcije $F$ je dana sa $rotF = (\partial_2 F_3 - \partial_3 F_2,\partial_3 F_1 - \partial_1 F_3, \partial_1 F_2 - \partial_2 F_1)$.
Dakle, imamo:
\begin{equation*}
rot\nabla f = (\partial_2(\partial_3 f) - \partial_3(\partial_2 f), \partial_3(\partial_1 f) - \partial_1(\partial_3 f), \partial_1(\partial_2 f) - \partial_2(\partial_1 f)).
\end{equation*}
Iz pretpostavke dovoljne glatkoće od $f$, imamo simetričnost Hesseove matrice, odnosno vrijedi
\begin{equation*}
\frac{\partial^2 f}{\partial x_i \partial x_j} = \frac{\partial^2 f}{\partial x_j \partial x_i}, \; \forall \; i, \; j \in \{1, 2, 3\}.
\end{equation*}
Sada je očito da je $rot \nabla f = 0$.
\end{document}