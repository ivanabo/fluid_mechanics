\documentclass[a4paper,oneside,12pt]{memoir} % jednostrano: promijeniti twoside u oneside

% Paket inputenc omogucava direktno unosenje hrvatskih dijakritickih znakova 
% opcija utf8 za unicode (unix, linux, mac)
% opcija cp1250 za windowse
\usepackage[utf8]{inputenc}  % ukoliko se koristi XeLaTeX onda je \usepackage{xunicode}\usepackage{xltxtra}

% Stil za diplomski, unutra je ukljucena podrska za hrvatski jezik
\usepackage{diplomski}
% bibliografija na hrvatskom
\usepackage[languagenames,fixlanguage,croatian]{babelbib} % zahtijeva datoteku croatian.bdf
% hiperlinkovi 
\usepackage[pdftex]{hyperref} % ukoliko se koristi XeLaTeX onda je \usepackage[xetex]{hyperref}

% Odabir familije fontova:
% koristenjem XeLaTeX-a mogu se koristiti svi fontovi instalirani na racunalu, npr
% \defaultfontfeatures{Mapping=tex-text}
% \setmainfont[Ligatures={Common}]{Hoefler Text}
% ili
% \newcommand{\nas}[1]{\fontspec{Adobe Garamond Pro}\fontsize{24pt}{24pt}\color{Chocolate}\selectfont #1}
% i onda \nas{Naslov ...}
\usepackage{txfonts} % times new roman 

\usepackage{amsmath, upgreek, amssymb}

\newtheorem{teorem}{Teorem}
\newtheorem{definicija}[teorem]{Definicija}
\newtheorem{nap}[teorem]{Napomena}

\newtheorem{problem}{Problem}
\newtheorem*{problem*}{Problem}
\newtheorem*{oznake}{Oznake}
\newtheorem{tvrdnja}[teorem]{Tvrdnja}

% Paket graphicx sluzi za manipuliranje grafikom 

%\usepackage[pdftex]{graphicx} % ukoliko se koristi XeLaTeX onda je \usepackage[xetex]{graphicx}

% Paket amsmath je vec ukljucen
% Dodatno definirane matematicke okoline:
% teorem (okolina: thm), lema (okolina: lem), korolar (okolina: cor),
% propozicija (okolina: prop), definicija (okolina: defn), napomena (okolina: rem),
% slutnja (okolina: conj), primjer (okolina: exa), dokaz (okolina: proof)
% Definirane su naredbe za ispisivanje skupova N, Z, Q, R i C
% Definirane su naredbe za funkcije koje se u hrvatskoj notaciji oznacavaju drukcije 
% nego u americkoj: tg, ctg, ... (\tgh za tangens hiperbolni)
% Takodjer su definirane naredbe za Ker i Im (da bi se razlikovala od naredbe za imaginarni dio kompleksnog
% broja, naredba se zove \slika).

\pagestyle{headings}
% uz paket fancyhdr mogu se lako kreirati fancy zaglavlja i podnozja
\pagestyle{empty}

\begin{document}

\subsubsection*{Mehanika fluida \\ Ivana Bobinac, Projektni zadatak: 2D optjecanje profila sa šiljkom}	
$ $
Problem koji proučavamo je sljedeći:

\begin{problem} \label{pr:1}
Stacionaran, bezvrtložan tok idealnog fluida oko profila sa šiljkom.
\end{problem}


\begin{definicija}
Ograničeno 1-povezano područje $P \subseteq \mathbb{R}^2$ nazivamo profil.
\end{definicija}

Nadalje će $P$ označavati profil koji ima šiljak s vanjskim otvorom $\beta \pi, \; 1 < \beta \leq 2,$ u nekoj točki $T$.


\begin{oznake}
\begin{flalign*}
\tilde{\Omega} & \; 1-povezano \; ograniceno \; podrucje \; t.d. \; P \subset \subset \tilde{\Omega}&&\\
\Omega &= \bar{\Omega} \backslash \bar{P} \; fluid, &&\\
S_1 &= \partial P \; granica \; profila, &&\\
S &= \partial \Omega, &&\\
S_0 &= S \backslash S_1 \; vanjska \; granica \; fluida, &&\\
\nu(x),&\; x \in S \; vanjska \; jedinicna \; normala.
\end{flalign*}
\end{oznake}


Matematička formulacija Problema \ref{pr:1} je sljedeća:

\begin{problem} \label{pr:2}
\[
	(P2) \left\{
		\begin{array}{ll}
			div v = 0 \; u \; \Omega, \\
			rot v = 0 \; u \; \Omega, \\
			v \cdot \nu = g \; na \; S,
		\end{array}
		\right.
\]
gdje je $g$ zadana funkcija za koju vrijedi $g_{\big|S_1} = 0$ i $\int_{\partial \Omega} g = 0$.
\end{problem}


Želimo preformulirati (P\ref{pr:2}) kao Neumannov problem. 

\begin{teorem}
Neka je $\Omega$ 1-povezano područje i neka je tok bezvrtložan. Tada je tok i potencijalan.
\end{teorem}
\begin{proof}
Dokazano na predavanjima.
\end{proof}

Kada bi $S$ bila glatka krivulja, imali bi jedinstveno rješenje $v = \nabla u$, gdje je $u$ rješenje Neumannove zadaće
  \[
    (**) \left\{
                \begin{array}{ll}
                  \Delta u = 0 \; u \; \Omega, \\
                  \frac{\partial u}{\partial \nu} = g \; na \; S.
                \end{array}
              \right.
  \]

No $S$ nije glatko (šiljak u nekoj točki $T$) te je stoga narušena regularnost rješenja.

Vidimo da uvjet $(**)_2$ nema smisla, no znamo da je ograničena funkcija $u$ rješenje zadaće $(**)$ ako i samo ako za svaku funkciju $w$ iz odgovarajućeg prostora test funkcija zadovoljava varijacijsku jednadžbu $(VF)$:
\begin{equation*}
\int_\Omega \nabla u \cdot \nabla w dS = \int_S gw ds.
\end{equation*}
$(VF)$ ima smisla i ona određuje slabo rješenje koje je jedinstveno, zadovoljava $(**)_1$ na $\Omega$ i rubni uvjet $(**)_2$ na $S\backslash \{T\}$.




\begin{tvrdnja} \label{tv:3}
Neka je $(r, \varphi)$ polarni sustav s polom u $T$ i s $x-osi$ koja tangira profil. U maloj okolini šiljka $T$ slabo rješenje ima sljedeći asimptotički oblik:
\begin{equation} \label{eq:1}
u(r, \varphi) = a r^{\frac{1}{\beta}} cos\frac{\varphi}{\beta} + Reg(r,\varphi)
\end{equation}
gdje je $a \neq 0$ konstanta, a $Reg(r, \varphi)$ označava funkciju kojoj je gradijent ograničen.
\end{tvrdnja}
\begin{nap}
Uočimo da se gradijent od $a r^{\frac{1}{\beta}} cos\frac{\varphi}{\beta}$ ponaša kao $r^{\frac{1}{\beta}-1}$. Zbog pretpostavke $1 < \beta \leq 2$, to je neograničeno; stoga $u \notin H^2(\Omega)$, ali vrijedi $u \in H^1(\Omega)$.
\end{nap}

\begin{proof}
Neka je $\Omega_0 = \{ (r, \varphi): 0 < r < r_0, 0 < \varphi < \beta \pi \}$ dovoljno mala okolina točke $T$. Tada $(VF)$ ima sljedeći oblik:
\begin{equation*}
\int_{\Omega_0} \nabla u \cdot \nabla w dS = \int_{\Gamma_0} gw ds,
\end{equation*}
gdje je $\Gamma_0 = S \cap \overline{\Omega_0}$.

\begin{nap}
\begin{equation*}
\nabla_{(r,\varphi)} = \frac{\partial}{\partial r} + \frac{1}{r} \frac{\partial}{\partial \varphi}
\end{equation*}
\end{nap}

Zbog gornje napomene slijedi da $(VF)$ ima oblik:
\begin{equation} \label{eq:2}
\int_0^{r_0} r \big[ \int_0^{\beta \pi} \Big( \frac{\partial u}{\partial r} \frac{\partial w}{\partial r} + \frac{1}{r^2} \frac{\partial u}{\partial \varphi} \frac{\partial w}{\partial \varphi} \Big) d\varphi \big] dr = \int_0^{r_0} \big[ g(r,0) w(r,0) + g(r, \beta \pi) w(r, \beta \pi) \big] dr.
\end{equation}

Za $r \in (0, r_0)$, možemo $u(r, \cdot)$ razviti u Fourierov red po kosinusima na intervalu $(0, \beta \pi)$, tj. po sustavu funkcija $\frac{2}{\beta \pi} cos \frac{k \varphi}{\beta}, k = 0, 1, ...$.

\begin{nap}
Neka je $f$ po dijelovima neprekidna funkcija na $[0, l]$. Proširimo funkciju $f$ po parnosti na $[-l, l]: f(-x) = f(x), x \in (0, l]$ i tom proširenju pridružimo Fourierov red: $b_n = 0, a_n = \frac{1}{l} \int_{-l}^l f(x) cos\frac{n \pi x}{l} dx = \frac{2}{l} \int_0^l f(x) cos\frac{n \pi x}{l} dx$.
\end{nap}

Dakle, imamo:
\begin{equation*}
u(r, \varphi) = \frac{2}{\beta \pi} \frac{u_0(r)}{2} + \sum_{k=1}^{\infty} u_k(r) \frac{2}{\beta \pi} cos \frac{k \varphi}{\beta},
\end{equation*}
gdje moramo odrediti $u_k(r), \; k = 0, 1, 2, ...$ \newline
Nadalje, uzmimo da je $w$ oblika
\begin{equation*}
w_j(r, \varphi) = \Psi(r) cos\frac{j \varphi}{\beta}, \; j = 0, 1, 2, ...
\end{equation*}
gdje je $\Psi$ funkcija koja se poništava u nekoj okolini točaka $r = 0$ i $r = r_0$.
Računamo:

\begin{align*}
\frac{\partial u}{\partial r} &= \frac{2}{\beta \pi} \frac{u_0'(r)}{2} + \sum_{k = 1}^{\infty} u_k'(r) \frac{2}{\beta \pi} cos \frac{k \varphi}{\beta}, \\
\frac{\partial u}{\partial \varphi} &= - \sum_{k = 1}^{\infty} u_k(r) \frac{2}{\beta \pi} \frac{k}{\beta} sin \frac{k \varphi}{\beta}, \\
\frac{\partial w_j}{\partial r} &= \Psi '(r) cos \frac{j \varphi}{\beta}, \\
\frac{\partial w_j}{\partial \varphi} &= - \frac{j}{\beta} \Psi (r) sin \frac{j \varphi}{\beta}
\end{align*}

Dalje računamo:
\begin{align*}
\frac{\partial u}{\partial r} \frac{\partial w_j}{\partial r} &= \Big( \frac{2}{\beta \pi} \frac{u_0'(r)}{2} + \sum_{k = 1}^{\infty} u_k'(r) \frac{2}{\beta \pi} cos \frac{k \varphi}{\beta} \Big) \Psi '(r) cos \frac{j \varphi}{\beta} \\
&= \frac{2}{\beta \pi} \frac{u_0'(r)}{2} \Psi '(r) cos \frac{j \varphi}{\beta} + \sum_{k = 1}^{\infty} u_k'(r) \Psi '(r) \frac{2}{\beta \pi} cos \frac{k \varphi}{\beta} cos \frac{j \varphi}{\beta}, \\
\frac{\partial u}{\partial \varphi} \frac{\partial w_j}{\partial r} &= \sum_{k = 1}^{\infty} u_k(r) \frac{2}{\beta \pi} \frac{k}{\beta} sin \frac{k \varphi}{\beta} \frac{j}{\beta} \Psi (r) sin \frac{j \varphi}{\beta} \\
&= \sum_{k = 1}^{\infty} \frac{k j}{\beta^2} u_k(r) \Psi (r) sin \frac{k \varphi}{\beta} sin \frac{j \varphi}{\beta}.
\end{align*}
Uvrštavanjem tih izraza u jednadžbu \ref{eq:2}, zbog ortogonalnosti funkcija $sin$ i $cos$, dobivamo:
\begin{equation*}
\int_0^{r_0} r u_j'(r) \Psi'(r) + \Big(\frac{j}{\beta} \Big)^2 \frac{1}{r} u_j(r) \Psi(r) = \int_0^{r_0} \Psi(r) [g(r,0) + (-1)^j g(r, \beta \pi)] dr.
\end{equation*}
Označimo $f_j(r) = g(r,0) + (-1)^j g(r, \beta \pi)$.
Parcijalnom integracijom, te zbog proizvoljosti funkcije $\Psi$ (iz Osnovne leme varijacijskog računa) imamo:
\begin{equation} \label{eq:3}
(r u_j'(r))' - \Big(\frac{j}{\beta} \Big)^2 \frac{1}{r} u_j(r) + f_j(r) = 0.
\end{equation}

\begin{nap}[I.Aganović, K.Veselić: Linearne diferencijalne jednadžbe]
Rješenja jednadžbe oblika $-(a(x) u'(x))' + b(x) u(x) = g(x)$, gdje su $a(x) = x r(x), \; r(0) \neq 0, \; b(x) = \frac{1}{x} s(x)$, $r$ i $s$ analitičke funkcije na $(0, l)$ i regularne u $x = 0$: \newline
Korijeni karakteristične jednadžbe su $\sigma_1 = \sigma, \; \sigma_2 = - \sigma$, gdje je $\sigma = \Big( \frac{s(0)}{r(0)} \Big)^{\frac{1}{2}}$. \newline
Ako $2 \sigma$ nije cijelo, homogena jednadžba $(aw')' - bw = 0$ ima na $(0,l)$ linearno nezavisna rješenja
\begin{align*}
w_1(x) &= x^\sigma (1 + c_1 x + ...), \\
w_2(x) &= x^{-\sigma} (1 + d_1 x + ...).
\end{align*}
Ako je $2 \sigma$ cijelo, tada su linearno nezavisna rješenja iste jednadžbe dana sa
\begin{align*}
w_1(x) &= x^\sigma (1 + c_1 x + ...), \\
w_2(x) &= c_0 w_1(x) lnx + x^{-\sigma} (d_0 + d_1 x + ...), \; d_0 \neq 0.
\end{align*}
Opće rješenje je dano sa $w(x) = A w_1(x) + B w_2(x), \; A, B \in \mathbb{R}$.
\end{nap}

U našem slučaju je
\begin{align*}
a(x) &= x r(x), \; r(x) = 1, \\
b(x) &= \frac{1}{x} s(x), \; s(x) = \Big( \frac{j}{\beta} \Big), \\
\sigma &= \frac{j}{\beta}.
\end{align*}

Dakle, opće rješenje homogene jednadžbe je oblika
  \[
    u_j^0(r) = \left\{
                \begin{array}{ll}
                  A_0 + B_0 lnr, \; j = 0, \\
                  A_j r^{\frac{j}{\beta}} + B_j r^{- \frac{j}{\beta}}, \; j = 1, 2, ...
                \end{array}
              \right.
  \]
gdje su $A_j, B_j, j = 0, 1, 2, ...$ konstante. \newline
Razvojem funkcije $f_j(r)$ u McLaurinov red $f_j(r) = \sum_{l=0}^\infty f_{jl} r^l$, te iz činjenice da je partikularno rješenje jednadžbe $(r u'(r))' - \Big( \frac{j}{\beta} \Big)^2 \frac{1}{r} u(r) + r^l = 0$ (npr. varijacijom konstanti) dano sa
  \[
    u_{jl}(r) = \left\{
                \begin{array}{ll}
                  \Big( \Big( \frac{j}{\beta} \Big)^2 - (l+1)^2 \Big)^{-1} r^{l+1}, \; l+1 \neq \frac{j}{\beta}, \\
                  - \frac{1}{2} (l+1)^{-1} r^{\frac{j}{\beta}} lnr, \; l+1 = \frac{j}{\beta},
                \end{array}
              \right.
  \]
dobivamo da je rješenje jednadžbe \ref{eq:3} dano sa
\begin{equation} \label{eq:4}
u_j(r) = u_j^0(r) + \sum_{l=0}^\infty f_{jl} u_{jl}(r).
\end{equation}
Iz uvjeta ograničenosti funkcije $u$ slijedi $B_j = 0, \; j = 0, 1, 2, ...$. Sada imamo:
\begin{equation} \label{eq:5}
u_j(r) = A_j r^{\frac{j}{\beta}} + \sum_{\substack{l=0 \\ l \neq \frac{j}{\beta} - 1}}^\infty f_{jl} \Big( \Big( \frac{j}{\beta} \Big)^2 - (l+1)^2 \Big)^{-1} r^{l+1} - \sum_{\substack{l=0 \\ l = \frac{j}{\beta} - 1}}^\infty f_{jl} \frac{\beta}{2j} r^{\frac{j}{\beta}} lnr, \; j = 0, 1, 2, ...
\end{equation}
Koeficijenti $A_j$ se određuju iz uvjeta da se $u_j(r)$ podudara s Fourierovim koeficijentom točnog rješenja $u$ zadane zadaće za dovoljno malo $r_0$: $u_j(r_0) = \frac{2}{\beta \pi} \int_0^{\beta \pi} u(r_0, \varphi) cos \frac{j \varphi}{\beta} d\varphi$.

U konačnici imamo
\begin{align}
u(r, \varphi) = \frac{A_0}{2} &+ \frac{2}{\beta \pi} \sum_{k=1}^\infty A_k r^{\frac{k}{\beta}} cos \frac{k \varphi}{\beta} - \sum_{l=1}^\infty \frac{f_{0l}}{2(l+1)^2} r^{l+1} \\ \notag
&+ \frac{2}{\beta \pi} \sum_{\substack{k > 0 \\ l \geq 0 \\ l \neq \frac{k}{\beta}-1}} f_{kl} \Big( \Big( \frac{k}{\beta} \Big)^2 - (l+1)^2 \Big)^{-1} r^{l+1} cos \frac{k \varphi}{\beta} \\ \notag
&- \frac{2}{\beta \pi} \sum_{k=1}^\infty \frac{\beta}{2k} r^{\frac{k}{\beta}} lnr cos \frac{k \varphi}{\beta} \sum_{l=\frac{k}{\beta}-1} f_{kl}
\end{align}
Stavimo li $a = \frac{2}{\beta \pi} A_1$, dobivamo upravo \ref{eq:1}.
\end{proof}
$ $ \newline
$ $ \newline















Vratimo se sada na rješavanje problema \ref{pr:2}. \newline
Pretpostavimo da je $g \in L^2(\Omega)$. \newline
Neka je $\Phi_0 : \Omega \rightarrow \mathbb{R}$ rješenje sljedeće zadaće:
\begin{problem} \label{pr:3}
  \[
    (P3) \left\{
                \begin{array}{ll}
                  \Delta \Phi_0 = 0 \; u \; \Omega, \\
                  \frac{\partial \Phi_0}{\partial \nu} = g \; na \; S.
                \end{array}
              \right.
  \]
\end{problem}

Definiramo $v_0 = \nabla \Phi_0$. \newline
Neka je $u : \Omega \rightarrow \mathbb{R}^2$ rješenje sljedeće zadaće:
\begin{problem} \label{pr:4}
\[
	(P4)_H \left\{
		\begin{array}{ll}
			div u = 0 \; u \; \Omega, \\
			rot u = 0 \; u \; \Omega, \\
			u \cdot \nu = 0 \; na \; S,
		\end{array}
		\right.
\]
\end{problem}

\begin{tvrdnja}
Neka je $v_0 \in L^2(\Omega) \cap C(\Omega)$ definirana kao gore, te neka je $u \in L^2(\Omega) \cap C(\Omega)$ rješenje $(P\ref{pr:4})_H$. Tada je svako rješenje Problema \ref{pr:1} dano sa:
\begin{equation*}
v = v_0 + u.
\end{equation*}
\end{tvrdnja}

\begin{oznake}
\begin{flalign*}
\Sigma & \; prerez \; od \; \Omega \; paralelan \; osi \; x \; koji \; prolazi \; kroz \; siljak, &&\\
\dot{\Omega} &= \Omega \backslash \Sigma, &&\\
\Sigma^+ & \; 'gornja' \; strana \; prereza, &&\\
\Sigma^- & \; 'donja' \; strana \; prereza.
\end{flalign*}
\end{oznake}

\begin{definicija}
Neka je $f : \dot{\Omega} \rightarrow \mathbb{R}$ proizvoljna funkcija. Za $y \in \Sigma$ definiramo
\begin{align*}
\Big( f\big|_{\Sigma^+} \Big) (y) &= \lim_{x \rightarrow y} f(x) \;\; (s \; 'gornje' \; strane), \\
\Big( f\big|_{\Sigma^-} \Big) (y) &= \lim_{x \rightarrow y} f(x) \;\; (s \; 'donje' \; strane),
\end{align*}
ako ti limesi postoje; \newline
te definiramo funkciju $[\;]$ na $\Sigma$ sa
\begin{equation*}
[f] = \Big( f\big|_{\Sigma^+} \Big) - \Big( f\big|_{\Sigma^-} \Big).
\end{equation*}
\end{definicija}

Neka je $c \in \mathbb{R}$ proizvoljna konstanta, te neka je $q : \dot{\Omega} \rightarrow \mathbb{R}$ rješenje sljedeće zadaće:
\begin{problem} \label{pr:5}
\[
	(P5) \left\{
		\begin{array}{ll}
			\Delta q = 0 \; u \; \dot{\Omega}, \\
			\frac{\partial q}{\partial \nu} = 0 \; na \; S, \\
			\big[q\big] = c, \\
			\Big[\frac{\partial q}{\partial \nu}\Big] = 0.
		\end{array}
		\right.
\]
\end{problem}

\begin{tvrdnja}
Neka je $u \in L^2(\Omega) \cap C(\Omega)$ rješenje $(P\ref{pr:4})_H$. Tada je $u$ nužno oblika $u = \nabla q$, gdje je $q$ rješenje $(P\ref{pr:5})$.
\end{tvrdnja}

Neka je $q_1 : \dot{\Omega} \rightarrow \mathbb{R}$ rješenje zadaće:
\begin{problem} \label{pr:6}
\[
	(P6) \left\{
		\begin{array}{ll}
			\Delta q_1 = 0 \; u \; \dot{\Omega}, \\
			\frac{\partial q_1}{\partial \nu} = 0 \; na \; S, \\
			\big[q_1\big] = 1, \\
			\Big[\frac{\partial q_1}{\partial \nu}\Big] = 0.
		\end{array}
		\right.
\]
\end{problem}

Tada je svako rješenje Problema \ref{pr:5} oblika $q = \gamma q_1$, gdje je $\gamma$ proizvoljna konstanta.

\begin{definicija}
Neka je $A \in IntP, \; A = (a_1, a_2)$ t.d. $A$ leži na pravcu koji sadrži $\Sigma$. Definiramo $\omega_1 : \mathbb{R}^2 \backslash \{A\} \rightarrow \mathbb{R}$ sa
\[
\omega_1 = \left\{
				\begin{array}{ll}
					\frac{1}{2 \pi} arctg(\frac{x-a_1}{y-a_2}), \; y > a_2, \\
					-\frac{1}{2} + \frac{1}{2 \pi} arctg(\frac{x-a_1}{y-a_2}), \; y < a_2.
				\end{array}
				\right.
\]
\end{definicija}

Neka je $\Phi_1 : \Omega \rightarrow \mathbb{R}$ rješenje sljedeće zadaće:
\begin{problem} \label{pr:7}
  \[
    (P7) \left\{
                \begin{array}{ll}
                  \Delta \Phi_1 = 0 \; u \; \Omega, \\
                  \frac{\partial \Phi_1}{\partial \nu} = -\frac{\partial \omega_1}{\partial \nu} \; na \; S.
                \end{array}
              \right.
  \]
\end{problem}

\begin{tvrdnja}
Neka je $\omega_1$ definiran kao gore, te neka je $\Phi_1 \in H^1(\Omega)$ rješenje $(P\ref{pr:7})$. Tada je svako rješenje Problema \ref{pr:6} oblika
\begin{equation*}
q_1 = \Phi_1 + \omega_1.
\end{equation*}
\end{tvrdnja}

Konačno, dobili smo da je rješenje Problema \ref{pr:1} oblika
\begin{equation*}
v = v_0 + u = \nabla \Phi_1 + u = \nabla \Phi_1 + \gamma \nabla q = \nabla (\Phi_0 + \gamma(\Phi_1 + \omega_1))
\end{equation*}

Iz Tvrdnje \ref{tv:3} slijedi da su $\Phi_0$ i $\Phi_1$ oblika
\begin{align*}
\Phi_0 &= a_0 r^{\frac{1}{\beta}} cos \frac{\varphi}{\beta} + Reg_0(r, \varphi), \\
\Phi_1 &= a_1 r^{\frac{1}{\beta}} cos \frac{\varphi}{\beta} + Reg_1(r, \varphi),
\end{align*}
gdje su $a_0 \neq 0$, $a_1 \neq 0$ konstante. \newline
Stoga je singularitet u brzini $v$ uzrokovan vodećim članom
\begin{equation} \label{eq:6}
(a_0 + \gamma a_1) r^{\frac{1}{\beta}} cos \frac{\varphi}{\beta}.
\end{equation}

\begin{tvrdnja}[Princip Kutta-Žukovskog] \label{tv:13}
U okolini šiljka brzina je ograničena.
\end{tvrdnja}

Iz Tvrdnje \ref{tv:3} sada slijedi da je $a_0 + \gamma a_1 = 0$.

\pagebreak

\subsubsection*{Literatura}
\begin{enumerate}
\item I. Aganović: Uvod u rubne zadaće mehanike kontinuuma, Element, 2003.
\item J. Tambača: Problem optjecanja profila sa šiljkom, diplomski rad, 1994.
\item I. Aganović, K. Veselić: Linearne diferencijalne jednadžbe, Element, 1997.
\end{enumerate}

\end{document}